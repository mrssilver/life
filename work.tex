\documentclass[17pt]{article}
% Allow the usage of UTF-8 characters
\usepackage[utf8]{inputenc}
% Allow the usage of graphics (.png, .jpg)
%\usepackage{graphicx}

\usepackage{geometry}
\geometry{top=17pt, 
bottom=72pt, 
left=20pt, 
right=70pt,
%textwidth=250pt, 
%textheight=600pt,
paperwidth=300pt,
paperheight=200pt

}

\usepackage{xcolor}
\definecolor{gold}{HTML}{999900}

\definecolor{milk}{HTML}{f0f0a9}

\definecolor{fnt}{HTML}{110000}

%\pagecolor{milk}
\color{fnt}
\usepackage{verbatim}

\usepackage{listings}

\lstset{
    extendedchars=true,
    literate={中}{{\textcolor{red}{中}}}1
}
%chinese
\lstdefinestyle{go}{
    language=Go,                % 语言设置为 Go
    basicstyle=\ttfamily\small, % 基本字体样式
    keywordstyle=\color{red},  % 关键字颜色
    commentstyle=\color{green}, % 注释颜色
    stringstyle=\color{red},    % 字符串颜色
    showstringspaces=false,     % 不显示字符串中的空格
    numbers=left,               % 行号在左侧
    numberstyle=\tiny\color{yellow}, % 行号样式
    stepnumber=1,               % 每行显示行号
    tabsize=4,                  % 缩进宽度
    breaklines=true,            % 自动换行
    backgroundcolor=\color{orange}, % 背景色(可选)
    frame=single,               % 边框样式
    rulecolor=\color{yellow},    % 边框颜色
    captionpos=b,               % 标题位置(bottom)
}
%define keywords add
\lstdefinelanguage{Go}{
    keywords=[1]{package, import, func, var, const, struct, interface},
    keywords=[2]{if, else, for, range, switch, case, default, defer, go, select},
    keywords=[3]{int, string, bool, float64, error, nil, true, false},
    sensitive=true,
    morecomment=[l]{//},       % 单行注释
    morecomment=[s]{/*}{*/},  % 多行注释
    morestring=[b]{"},        % 字符串
    morestring=[b]{`},        % 原始字符串(raw string)
}

% Start the document
\begin{document}

% Create a new 1st level heading
\section{Main Heading}

\begin{verbatim}
#include <stdio.h>
int main() {
    printf("Hello, World!\n");
    return 0;
}
\end{verbatim}

%\verbatiminput{code.py}
%read code.py
\verb |\# to #|

\begin{comment}
这部分内容不会在最终文档中显示,
用于临时注释大段文本。verbatim function 
\end{comment}

Go 的 \lstinline[language=Go]|fmt.Println("Hello")| 是打印函数。

\begin{lstlisting}[style=go]
package main

import "fmt"

func main() {
    fmt.Println("Hello, World!")
}
\end{lstlisting}

%\lstinputlisting[style=go]{hello.go} %read file

\begin{lstlisting}[language=Python]
def factorial(n):
    if n == 0:
        return 1
    else:
        return n * factorial(n-1)
\end{lstlisting}

\textcolor{gold}{
Here's a line-by-line explanation of the weather-process-data function:
}
(defun weather-process-data (city data)
  "处理并存储天气数据"
  ;; Function definition: weather-process-data takes two parameters:
  ;; - city: the name of the city being processed
  ;; - data: the weather data from API response
  
  (condition-case err
  ;; Begin error handling: if any error occurs in the following code,
  ;; the error will be caught and handled by the error section at the end
  
      (let* ((city-info (gethash city weather-city-data))
      ;; Retrieve existing weather data for this city from the hash table
      
             (existing-hourly (or (plist-get city-info :hourly) '()))
             ;; Extract existing hourly data, or use empty list if none exists
             
             (new-hourly (mapcar jing'weather--create-plist 
                                (alist-get 'list data)))
             ;; Convert new API data to property lists using weather--create-plist
             ;; The API data is found in the 'list field of the response
             
             (all-hourly (append existing-hourly new-hourly))
             ;; Combine existing and new hourly data
             
             (filtered-hourly (weather--filter-by-time-range all-hourly 1 5)))
             ;; Filter data to include only past 1 day and next 5 days
             
        ;; Sort the filtered hourly data by timestamp (dt field)
        (setq filtered-hourly (cl-sort filtered-hourly \#'< :key (lambda (x) (plist-get x :dt))))
        
        ;; Remove duplicate entries based on timestamp
        (setq filtered-hourly (cl-remove-duplicates filtered-hourly 
                                                   :test (lambda (a b) 
                                                          (= (plist-get a :dt) 
                                                             (plist-get b :dt))))))
        
        ;; Group the filtered data by date
        (let ((daily-data (weather--group-by-date filtered-hourly)))
          ;; Store the processed data in the hash table
          (puthash city `(:city ,city 
                         :hourly ,filtered-hourly 
                         :daily ,daily-data
                         :last-update ,(current-time))
                   weather-city-data)))
    ;; End of main processing code
    
    (error 
     ;; Error handling section: executed if any error occurred above
     ;; Display a message about the processing failure but don't store the error
     (message "%s: 数据处理失败 - %s" city (error-message-string err)))))
     ;; Format: "City: 数据处理失败 - Error message"


This function:

1. Takes a city name and API response data as input

2. Uses error handling to catch any processing issues

3. Combines existing data with new API data

4. Filters the data to a specific time range (past 1 day, next 5 days)

5. Sorts and removes duplicates from the data

6. Groups the hourly data by date

7. Stores the processed data in a hash table for later use

8. Shows an error message if processing fails, but doesn't store the error for display in the final report

The function ensures that only valid, properly formatted weather data gets stored and displayed, while any processing errors are simply logged as messages without affecting the overall report.

% Uncomment the following two lines if you want to have a bibliography
%\bibliographystyle{apalike}
%\bibliography{document}

\end{document}